\documentclass[11pt]{article}
\usepackage{colacl}
\sloppy

\title{COMP90049 Project II Report}
\author{Clement Poh cjpoh 269508}

\begin{document}
\maketitle


\begin{abstract}

Text categorisation is a machine learning process that attempts to classify text
documents in to different categories or genres. This paper explores the process
of text categorisation from feature extraction to classification with random
forests and a cursory attempt using support vector machines.

\end{abstract}


\section{Introduction}

Document classification and text categorisation is the task of classifying text
documents into a set of predefined categories. Traditionally classification was
in the province of library sciences. With the advent of computer systems and the
internet and the subsequent explosion of information, methods of automatically
categorising the information effectively are an active field of research in
computer science. Modern applications of categorisation are used searching and
spam filtering.

In this report, modern concepts from computer science are applied to a problem
already solved a long ago in library science; the classification of classics.

\section{Aim}

The aim of this report is to develop a text categorisation system to classify a
collection of books. The system is provided with three hundred books to be
trained on, to generate a feature set that is used to predict the categories of
another hundred books.

\section{Metrics}


\section{Approach}

The system is split into four parts. The first two parts are responsible for
feature extraction and selection. The latter two are used for classification.

The first extracts tokens and other data from the training documents. It
analyses the data, then outputs the training vectors for the classifiers. 

The second creates the feature vectors for the test data from the books
themselves and according to the analysis carried out by the previous part.



\subsection{Feature Extraction & Selection}



\subsection{Categorisation}

% Text of the subsection with citations such as 
% \newcite{Spa72}, \newcite{Kay86} and \newcite{MosWal64} 

 
\subsubsection{Subsubsection}

% Text of the subsubsection (see Table~\ref{table1}).

\begin{comment}
    \begin{table}[h]
     \begin{center}
        \begin{tabular}{|l|l|}

              \hline
              Corpus & Features\\
              \hline\hline
              AAA & 1M words\\
              BBB & spoken corpus (expensive)\\
              CCC & 2M words\\
                & free (to academics)\\
              \hline

        \end{tabular}
    \caption{The caption of the table}\label{table1}
     \end{center}
    \end{table}
\end{comment}




\section{Section}

\section{Section}

\section{Conclusions}

Concluding text.

\bibliographystyle{acl}
\bibliography{sample}

\end{document}
